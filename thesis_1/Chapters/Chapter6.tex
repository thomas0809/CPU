% Chapter Template

\chapter{实验总结} % Main chapter title

\label{Chapter6} % Change X to a consecutive number; for referencing this chapter elsewhere, use \ref{ChapterX}

%----------------------------------------------------------------------------------------
%	SECTION 1
%----------------------------------------------------------------------------------------

\section{实验中遇到的问题}

%-----------------------------------
%	SUBSECTION 1
%-----------------------------------
\subsection{教学软件使用不当}

由于我们开始造计算机比较早,当我们写好代码开始调试时,发现网络学堂上还没有放出flashandram软件。我们通过助教得到了这个软件,但是不会使用flashandram和term软件,我们仔细阅读实验参考书和ppt等参考资料,没有发现如何使用这些软件的介绍。通过与学长交流和向助教询问,才知道flashandram软件使用前要将impact加入到系统path中,在运行term时要出入串口端口号。

%-----------------------------------
%	SUBSECTION 2
%-----------------------------------

\subsection{最初几条指令会出错}
我们在基础部分代码写完之后,经过简单的上机调试,在12.5M时钟下运行监控程序,很快就能正确输出OK信息,而使用25M时钟之后,发出的信息出现了错误。为了解决这个问题,我们花了很长时间的调试,一条一条指令、一个阶段一个阶段输出调试信息,判断是否符合预期。在手动时钟、或者频率低一些的时钟下,所有指令都是正确运行的。因为也不是很清楚为什么25M时钟下会有这些问题,不是很好判断是否是哪方面的延迟造成的问题,我们开始尝试各种方法试图减少访存等延迟,但是还是没有能解决问题。

调试这个问题一两天之后,在一次偶然的尝试下,我们在监控程序开始执行之前插入了20个气泡,然后输出信息就正常了。这个现象可能的原因是FPGA刚烧入程序就开始运行,可能最开始的时候电路还没有稳定,或者烧入没有完全完成,所以结果会出现错误。因此我们最后都在开始执行监控之前等待20个周期,然后再开始执行。

%-----------------------------------
%	SUBSECTION 3
%-----------------------------------

\subsection{中断处理}

中断的检测与处理是在IF\_ID段间锁存器处理,根据不同的中断信号来判断CPU进入到哪个状态。我们最开始的写法是在上升沿到来后判断中断信号,再进行后面的各种处理。但是在实验中发现,如果在上升沿到来后进行中断判断,并进行处理,延迟很大,使得指令和PC的值会出错。所以我们将中断的判断部分变为了组合逻辑,在上升沿到来时仅输信号,减小延迟

中断处理的另一个问题在于,我们做ESC硬件中断、软件中断时,需要在中断处理过后跳回到中断发生时的PC,这时IF\_ID段间锁存器保留的PC为当前指令的下一条PC,所以我们需要将PC减一再保存到栈中。但是,在跳转指令发生时这样处理会出问题。比如以下的例子:

\textbf{\textit{Example6}}: \quad B 0xFF; \quad NOP; \quad JR R7

在上面的例子中,我们写了一个死循环程序。但是如果在延迟槽NOP语句发生中断,那么IF\_ID段间锁存器的PC为JR R7的PC值,这时PC减一存到栈中,那么跳回后将继续执行NOP,然后JR R7跳出用户程序。这样使得中断后跳出了死循环,与我们预期不符。所以如果中断发生在延迟槽时,我们不能将PC减一,而是应该将PC减二,下一次继续执行跳转指令。

%-----------------------------------
%	SUBSECTION 4
%-----------------------------------
\subsection{B指令的跳转偏移量}

我们在修改监控程序将机器码转为汇编码的部分,由于代码很长,使用B、BNEZ、BEQZ等指令时遇到了偏移地址位数不足的情况。在意识到这个问题之前,我们监控程序运行的PC变化混乱。所以,我们在需要B类跳转时,首先跳到一个比较近的偏移地址,然后通过J指令跳到目的地址,因为J指令的跳转地址为16位,所以可以解决这个问题。

%-----------------------------------
%	SUBSECTION 5
%-----------------------------------
\subsection{多道程序}

在实现多道程序时,之前讨论的原理看起来很简单,但是调试花了我们很长的时间。我们在修改了中断延迟、保存现场的一些问题之后,多道程序还是不能正常地运行。现象为键盘VGA终端可以正常执行,但是电脑端term会出现卡死的情况。由于两套代码基本一致,所以不会出现跳回PC不对的问题。我们仔细分析这个问题,才发现了问题的根源。

这个问题产生的原因在于我们时钟中断发生的频率较低,隔50000个周期才切换一次。这时,如果在电脑端通过term输入某条指令,但是CPU不在运行电脑端的监控程序,这时我们需要将之前的输入保存,等待CPU访问。但是串口CPLD不具备缓冲区的功能,所以输入的指令就被丢弃,导致term端进入死循环,而CPU在等待输入。所以我们将切换周期变快,隔1000个周期切换一次,这时两套监控程序可以正常运行。

%----------------------------------------------------------------------------------------
%	SECTION 2
%----------------------------------------------------------------------------------------

\section{实验感想}


%-----------------------------------
%	SUBSECTION 1
%-----------------------------------
\subsection{对CPU的理解}

通过这次亲自动手编写CPU的实验,我们深入地理解了一个实现流水的CPU有哪些结构,各个结构应该实现什么功能。对于我们之前一直难以理解的冲突处理部分,在我们解决了数据冲突、控制冲突之后,我也有了更透彻的理解,学习到了旁路转发是如何发挥作用的。深入地理解了中断的原理和处理中断的程序,虽然我们做的中断处理很简单,但是通过实验中的各种问题,也提升了我们对中断的理解。

%-----------------------------------
%	SUBSECTION 2
%-----------------------------------
\subsection{提高解决问题的能力}

实验中我们遇到了各种问题,基础版的CPU调试工作还好,我们可以通过手动时钟等各种办法调试,但是当调试多道程序时,我们不能通过手动时钟的方式,因为中断发生一次就要上千个周期。所以我们不是盲目地尝试与实验,而是仔细观察现象、分析问题发生的原因、进而解决问题,提高了解决问题的能力。

总之在这个实验中,出现的问题千奇百怪,原因也各种各样,我们需要动用 各种资源,做出各种猜想来尝试解决问题,实验过后我们解决问题的能力得到了 大大提高。

%-----------------------------------
%	SUBSECTION 3
%-----------------------------------
\subsection{乐观自信的态度}

在实现CPU的过程中,很多问题花了我们很长的时间调试,包括中断处理、多道程序,都花了我们很长的时间,中间调试过程中也有了很多次放弃的念头。但是我们相信我们一定可以克服困难,解决问题,通过艰苦不断地努力,虽然我们没有刷夜,但是我们还是费了很大的精力调试成功。

对于一些组同学的时钟可以达到50M,我们很佩服他们。因为在我们实验开始,我们从助教同学中听说最快的时钟频率只能达到25M,所以我们将CPU调到25M后就没有再进行尝试。等到我们做完多道程序时,才知道班里同学做出了50M的CPU,这时我们再想修改最早的代码,尝试50M时,我们已经力不从心了。所以在实验时要敢想敢做,不断尝试,没有什么事情不可能做到。



%----------------------------------------------------------------------------------------
%	SECTION 3
%----------------------------------------------------------------------------------------

\section{教学建议}


%-----------------------------------
%	SUBSECTION 1
%-----------------------------------
\subsection{对实验环境的建议}

对于这次实验的硬件条件和软件资源,我们有一些建议。首先板子上的调试的LED灯太少,导致我们不能很好地观察现象,增加了调试难度。所以希望可以有更多的灯以供调试。另一点是烧入程序的线很不稳定,导致烧入程序浪费了很多时间。教学软件的问题浪费了我们很长的时间,所以希望以后可以有一个详细的软件说明,方便我们使用。

\subsection{评分制度}

对于班内竞争的评分方式,可能在很多的年级比较合理,但是今年我们班竞争十分激烈。据我们了解,在我们做完键盘VGA版的时候,其他班级才刚刚开始写基础版的CPU,但是我们班各组的进度都很快。这种方式导致我们班工作压力都很大,需要做很多的扩展功能才能在班级中立足。所以希望仿照软件工程的方式,在各班的组中随机分配。

\subsection{实验内容}

基础版本的CPU可以考虑增加难度,其实在实验过程中,设计数据通路和控制信号的工作很好地完成后,后续写代码的工作量很小。有的组调试一天就可以正常运行了。所以希望可以增加一些难度,比如中断处理可以作为基础功能。

\subsection{感谢}

最后感谢各位老师的教导和助教的帮助,在我们编写CPU的过程中从老师和助教吸取了很多的经验和指导,对于我们解决实验中的问题很有帮助。同时要感谢班级中与我们共同努力的各组,我们调试程序时也与同学进行了很多的讨论。


