% Chapter Template

\chapter{实验总结} % Main chapter title

\label{Chapter6} % Change X to a consecutive number; for referencing this chapter elsewhere, use \ref{ChapterX}

计算机组成原理的课程实验是本学期的一个重要项目。我们很成功地完成了这个实验,有很大的收获。

实验开始之前我们使用了较长一段时间进行了全面而细致的实验设计,包括详细的数据通路、控制信号设计,这使得我们之后实现过程中能够做到思路清晰、目标明确,大大地简化了实现的复杂度。我们在设计中充分考虑了各种数据冲突、控制冲突等情况,并进行了妥善的处理。

从实验开始经过大约八九天的时间,我们就完成了实验基础功能的实现和调试,之后进行了很多的扩展功能的实现。我们实现的扩展包括多种软硬件中断、键盘输入(包括汇编代码转机器码)、VGA显示(包括机器码反汇编成汇编代码)、屏幕保护界面、多道程序分时操作等。

这个实验并不像传说中的那么辛苦,虽然我们在其中也遇到了很多问题,但是经过认真的分析和耐心的调试,最终都得到了很好的解决。我们对我们做出的成果也是比较满意的。接下来我们介绍一下我们实验中遇到的一些问题,此次实验的收获以及一些课程的建议。

%----------------------------------------------------------------------------------------
%	SECTION 1
%----------------------------------------------------------------------------------------

\section{实验中遇到的问题}

%-----------------------------------
%	SUBSECTION 1
%-----------------------------------
\subsection{教学软件使用不当}

由于我们开始造计算机比较早,当我们写好代码开始调试时,发现网络学堂上还没有放出flashandram软件。我们通过助教得到了这个软件,但是不会使用flashandram和term软件,我们仔细阅读实验参考书和ppt等参考资料,没有发现如何使用这些软件的介绍。通过与学长交流和向助教询问,才知道flashandram软件使用前要将impact加入到系统path中,在运行term时要出入串口端口号。

%-----------------------------------
%	SUBSECTION 2
%-----------------------------------

\subsection{最初几条指令会出错}
我们在基础部分代码写完之后,经过简单的上机调试,在12.5M时钟下运行监控程序,很快就能正确输出OK信息,而使用25M时钟之后,发出的信息出现了错误。为了解决这个问题,我们花了很长时间的调试,一条一条指令、一个阶段一个阶段输出调试信息,判断是否符合预期。在手动时钟、或者频率低一些的时钟下,所有指令都是正确运行的。因为也不是很清楚为什么25M时钟下会有这些问题,不是很好判断是否是哪方面的延迟造成的问题,我们开始尝试各种方法试图减少访存等延迟,但是还是没有能解决问题。

调试这个问题一两天之后,在一次偶然的尝试下,我们在监控程序开始执行之前插入了20个气泡,然后输出信息就正常了。这个现象可能的原因是FPGA刚烧入程序就开始运行,可能最开始的时候电路还没有稳定,或者烧入没有完全完成,所以结果会出现错误。因此我们最后都在开始执行监控之前等待20个周期,然后再开始执行。

%-----------------------------------
%	SUBSECTION 3
%-----------------------------------

\subsection{中断处理}

中断的检测与处理是在IF\_ID段间锁存器处理,根据不同的中断信号来判断CPU进入到哪个状态。我们最开始的写法是在上升沿到来后判断中断信号,再进行后面的各种处理。但是在实验中发现,如果在上升沿到来后进行中断判断,并进行处理,延迟很大,使得指令和PC的值会出错。所以我们将中断的判断部分变为了组合逻辑,在上升沿到来时仅输信号,减小延迟

中断处理的另一个问题在于,我们做ESC硬件中断、软件中断时,需要在中断处理过后跳回到中断发生时的PC,这时IF\_ID段间锁存器保留的PC为当前指令的下一条PC,所以我们需要将PC减一再保存到栈中。但是,在跳转指令发生时这样处理会出问题。比如以下的例子:

\textbf{\textit{Example6}}: \quad B 0xFF; \quad NOP; \quad JR R7

在上面的例子中,我们写了一个死循环程序。但是如果在延迟槽NOP语句发生中断,那么IF\_ID段间锁存器的PC为JR R7的PC值,这时PC减一存到栈中,那么跳回后将继续执行NOP,然后JR R7跳出用户程序。这样使得中断后跳出了死循环,与我们预期不符。所以如果中断发生在延迟槽时,我们不能将PC减一,而是应该将PC减二,下一次继续执行跳转指令。

%-----------------------------------
%	SUBSECTION 4
%-----------------------------------
\subsection{B指令的跳转偏移量}

我们在修改监控程序将机器码转为汇编码的部分,由于代码很长,使用B、BNEZ、BEQZ等指令时遇到了偏移地址位数不足的情况。在意识到这个问题之前,我们监控程序运行的PC变化混乱。所以,我们在需要B类跳转时,首先跳到一个比较近的偏移地址,然后通过J指令跳到目的地址,因为J指令的跳转地址为16位,所以可以解决这个问题。

%-----------------------------------
%	SUBSECTION 5
%-----------------------------------
\subsection{汇编与反汇编}

我们实现了键盘输入和VGA输出的扩展,模仿实现了Term程序的界面。这个功能的实现看似普通,但是中间就会有很多不好解决的问题。

Term程序是在PC端用C++语言实现的,因此一些功能是用高级语言来进行处理,就会很方便。比如输入的命令A、U等如何转成给CPU的输入,还要支持A 4000这样带参数的输入。尤其是汇编和反汇编的过程,通过硬件的实现就要比高级语言麻烦很多。也有一些替代的方法,比如放弃完全模仿Term,直接使用机器码的二进制或十六进制输入输出,但是我们精益求精,还是克服困难实现了汇编和反汇编。

汇编相对简单一些,对一行汇编代码,在输入的过程中硬件缓存这一行的数据,用户按回车之后根据这一行的数据判断翻译成机器码。这个虽然麻烦,但是多写一些判断语句就可以实现。

反汇编就要麻烦一点,反汇编阶段CPU输出的是指令机器码,一个指令对应一段汇编代码,长度还不一定固定。VGA的实现是一个字符一个字符接收、显示。如果在硬件实现反汇编,就需要每个指令建立一个状态机,根据指令依次给VGA发送字符,这样实现起来就非常麻烦。我们为了怎么实现这个功能还争执了一段时间。最后我们找到了一个替代的方案,即修改监控程序,在反汇编过程中由监控程序把机器码翻译成字符,再发送给VGA。这样其实也很麻烦,需要写很多汇编代码,而且这时候发现有的情况16位指令B指令立即数的范围就不太够了,需要特殊处理。但是这样通过汇编语言的实现,应该还是要比直接用硬件语言处理方便一些。

%-----------------------------------
%	SUBSECTION 6
%-----------------------------------
\subsection{多道程序}

在实现多道程序时,之前讨论的原理看起来很简单,但是调试花了我们很长的时间。我们在修改了中断延迟、保存现场的一些问题之后,多道程序还是不能正常地运行。现象为键盘VGA终端可以正常执行,但是电脑端term会出现卡死的情况。由于两套代码基本一致,所以不会出现跳回PC不对的问题。我们仔细分析这个问题,才发现了问题的根源。

这个问题产生的原因在于我们时钟中断发生的频率较低,隔50000个周期才切换一次。这时,如果在电脑端通过term输入某条指令,但是CPU不在运行电脑端的监控程序,这时我们需要将之前的输入保存,等待CPU访问。但是串口CPLD不具备缓冲区的功能,所以输入的指令就被丢弃,导致term端进入死循环,而CPU在等待输入。所以我们将切换周期变快,隔1000个周期切换一次,这时两套监控程序可以正常运行。

%----------------------------------------------------------------------------------------
%	SECTION 2
%----------------------------------------------------------------------------------------

\section{实验感想}


%-----------------------------------
%	SUBSECTION 1
%-----------------------------------
\subsection{对CPU的理解}

通过这次亲自动手编写CPU的实验,我们深入地理解了一个实现流水的CPU有哪些结构,各个结构应该实现什么功能。对于我们之前一直难以理解的冲突处理部分,在我们解决了数据冲突、控制冲突之后,我也有了更透彻的理解,学习到了旁路转发是如何发挥作用的。深入地理解了中断的原理和处理中断的程序,虽然我们做的中断处理很简单,但是通过实验中的各种问题,也提升了我们对中断的理解。

%-----------------------------------
%	SUBSECTION 2
%-----------------------------------
\subsection{提高解决问题的能力}

实验中我们遇到了各种问题,基础版的CPU调试工作还好,我们可以通过手动时钟等各种办法调试,但是当调试多道程序时,我们不能通过手动时钟的方式,因为中断发生一次就要上千个周期。所以我们不是盲目地尝试与实验,而是仔细观察现象、分析问题发生的原因、进而解决问题,提高了解决问题的能力。

总之在这个实验中,出现的问题千奇百怪,原因也各种各样,我们需要动用 各种资源,做出各种猜想来尝试解决问题,实验过后我们解决问题的能力得到了 大大提高。

%-----------------------------------
%	SUBSECTION 3
%-----------------------------------
\subsection{乐观自信的态度}

在实现CPU的过程中,很多问题花了我们很长的时间调试,包括中断处理、多道程序,都花了我们很长的时间,中间调试过程中也有了很多次放弃的念头。但是我们相信我们一定可以克服困难,解决问题,通过艰苦不断地努力,虽然我们没有刷夜,但是我们还是费了很大的精力调试成功。

对于一些组同学的时钟可以达到50M,我们很佩服他们。因为在我们实验开始,我们根据最开始的访存时序设计,认为在这一实验中最快的时钟频率只能达到25M,所以我们将CPU调到25M后就没有再进行尝试。等到我们做完多道程序时,班里有同学做出了50M的CPU,这时我们再想修改最早的代码,尝试50M时,经过探索,虽然能够输出OK,后续指令出现了错误,最后没有成功。所以在实验时要敢想敢做,不断尝试,不要被一些主观的想法所局限。


\subsection{联系前期课程}

计算机组成原理实验其实和数字逻辑设计实验有很大的相似性。我们(钱雨杰、董胤蓬)去年数字逻辑设计课程实现了一个Core Ball的游戏,在此次实验中,我们将以前的实验内容经过精简后加入本次的项目,以一个很新颖的方式呈现出来——屏幕保护程序。

屏幕保护是现在一般PC都会有的一个功能,在硬件上也很简单,一段时间没有输入切换就可以(虽然PC应该不是在硬件上做的这个事情)。我们去掉之前项目中的游戏功能,仅保留动画效果,做成一个屏幕保护界面,成为此次项目的一个功能。这样的尝试让我们将以前的课程和工作结合到此次实验中来,并取得了不错的效果。


%----------------------------------------------------------------------------------------
%	SECTION 3
%----------------------------------------------------------------------------------------

\section{教学建议}


%-----------------------------------
%	SUBSECTION 1
%-----------------------------------
\subsection{对实验环境的建议}

对于这次实验的硬件条件和软件资源,我们有一些建议。首先板子上的调试的LED灯太少,导致我们不能很好地观察现象,增加了调试难度。经常需要不断更改LED灯显示的信号,每次改了要重新编译,ISE的编译还很慢。所以希望可以有更多的LED灯或其他设备以供调试。另一点是烧入程序的线很不稳定,导致烧入程序浪费了很多时间。教学软件的问题浪费了我们很长的时间,所以希望以后可以有一个详细一些的软件、硬件方面的说明,方便我们使用。

\subsection{评分制度}

对于班内竞争的评分方式,可能在很多的年级比较合理,但是今年我们班竞争比较激烈。据我们了解,在我们做完键盘VGA版的时候,其他班级才刚刚开始写基础版的CPU,而我们班各组的进度都很快。这种方式导致我们班工作压力都很大,大家都把很多扩展功能当成基础功能来做了。所以按班级分组、评分其实意义不是很大,也不一定合理。

\subsection{实验内容}

基础版本的CPU可以考虑增加难度,其实在实验过程中,设计数据通路和控制信号的工作很好地完成后,后续写代码的工作量很小。有的组调试一天就可以正常运行了。所以希望可以增加一些难度,比如中断处理可以作为基础功能。

\subsection{感谢}

最后感谢老师的指导和助教、学长的帮助,在我们实验实现的过程中,老师的指导和助教、学长的经验、意见,对于我们解决实验中的问题很有帮助。同时要感谢班级中与我们共同努力的各组,我们调试程序时也与同学进行了很多的讨论。


