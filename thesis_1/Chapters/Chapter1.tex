% Chapter 1

\chapter{实验目的} % Main chapter title

\label{Chapter1} % For referencing the chapter elsewhere, use \ref{Chapter1} 

%----------------------------------------------------------------------------------------

% Define some commands to keep the formatting separated from the content 
\newcommand{\keyword}[1]{\textbf{#1}}
\newcommand{\tabhead}[1]{\textbf{#1}}
\newcommand{\code}[1]{\texttt{#1}}
\newcommand{\file}[1]{\texttt{\bfseries#1}}
\newcommand{\option}[1]{\texttt{\itshape#1}}

本实验是清华大学计算机科学与技术系开设的《计算机组成原理》课程实验。

实验任务是设计和实现一台16位支持指令流水的计算机。实验计算机的CPU采用五段流水线结构,支持THCO MIPS指令系统,并适当应用数据旁路、分支预测等技术提高流水线效率;使用SRAM作为存储器,并对内存进行管理;实验计算机的I/O通过串口与运行终端程序的PC连接,实现输入输出。实验计算机实现后需要运行监控程序,并可以通过监控程序实现用户写入指令、执行指令、查看寄存器和内存等操作。

实验可以在基础要求上进行一些扩展,包括软硬件中断处理、PS2键盘输入、VGA输出、双机通信、多道程序等。

通过实验,可以加深对于计算机系统知识的理解,进一步理解和掌握流水线结构计算机的各部件组成和内部工作原理,掌握计算机外部设备输入输出的设计实现,培养硬件设计和调试的能力。

%----------------------------------------------------------------------------------------

