% Chapter Template

\chapter{模块划分与接口设计} % Main chapter title

\label{Chapter4} % Change X to a consecutive number; for referencing this chapter elsewhere, use \ref{ChapterX}

%----------------------------------------------------------------------------------------
%	SECTION 1
%----------------------------------------------------------------------------------------

\section{IFetch}
RAM2读写指令模块,时序逻辑。同时根据输入PC和暂停流水线信号输出下一条顺序执行指令的PC值。

\begin{table}[H]
\begin{center}
\renewcommand{\arraystretch}{1.3}
\small
\caption{IFetch Interface}
\label{tab:treatments}
\begin{tabular}{|p{3cm}<{\centering}|p{1.4cm}<{\centering}|p{7cm}<{\centering}|}
\hline
接口名称 & 类型 & 功能 \\
\hline
clk & in & \\
\hline
Ram2Addr & out & 0000 \\
\hline
Ram2Data & inout & \\
\hline
PC & in & \\
\hline
PCInc & out & \\
\hline
IR & out & \\
\hline
PCStop & in & \\
\hline
IFWE & in & \\
\hline
IFData & in &  \\
\hline
IFAddr & in & \\
\hline
Ram2EN & out & \\
\hline
Ram2WE & out & \\
\hline
Ram2OE & out & \\
\hline
\end{tabular}
\end{center}
\end{table}

%----------------------------------------------------------------------------------------
%	SECTION 2
%----------------------------------------------------------------------------------------

\section{IF$\_$ID$\_$REGISTER}

IF$\_$ID段间锁存器,时序逻辑。除了传递信号外,还需处理中断,通过硬件方式保存现场并跳到中断处理程序。另外,还需要根据Forward Unit模块产生的插气泡信号和MEM产生的写RAM2信号进行停止流水线操作。

\begin{table}[H]
\begin{center}
\renewcommand{\arraystretch}{1.3}
\small
\caption{IF$\_$ID$\_$REGISTER Interface}
\label{tab:treatments}
\begin{tabular}{|p{3cm}<{\centering}|p{1.4cm}<{\centering}|p{7cm}<{\centering}|}
\hline
接口名称 & 类型 & 说明 \\
\hline
clk & in & 时钟信号 \\
\hline
INPC & in & 输入PC值 \\
\hline
INIR & inout & 输入指令值 \\
\hline
INTERRUPT & in & 中断信号,包括四种中断 \\
\hline
IFIDSTOP & in & 插入气泡信号 \\
\hline
IF$\_$FLUSH & in & 跳转指令清除延迟槽信号 \\
\hline
IF$\_$WRITE$\_$RAM2 & in & Ram2写指令,暂停流水线 \\
\hline
OUTPC & out & 输出PC值 \\
\hline
OUTIR & out &  输出指令值\\
\hline
\end{tabular}
\end{center}
\end{table}

%----------------------------------------------------------------------------------------
%	SECTION 3
%----------------------------------------------------------------------------------------

\section{controller}

CPU控制器,组合逻辑。译码指令来产生控制信号。

\begin{table}[H]
\begin{center}
\renewcommand{\arraystretch}{1.3}
\small
\caption{controller Interface}
\label{tab:treatments}
\begin{tabular}{|p{3cm}<{\centering}|p{1.4cm}<{\centering}|p{7cm}<{\centering}|}
\hline
接口名称 & 类型 & 说明 \\
\hline
INSTRUCTION & in & 需要译码的指令 \\
\hline
SRCREG1 & out & 第一个源寄存器编号 \\
\hline
SRCREG2 & out & 第二个源寄存器编号\\
\hline
TARGETREG & out & 目的寄存器编号 \\
\hline
EXTENDIMM & out & 扩展后的立即数 \\
\hline
ALUOP & out & ALU操作符 \\
\hline
ALUSRCA & out & ALU第一个操作数选择信号 \\
\hline
ALUSRCB & out & ALU第二个操作数选择信号 \\
\hline
MEMTOREG & out &  读内存信号 \\
\hline
REGWRITE & out & 写寄存器信号 \\
\hline
MEMWRITE & out & 写内存信号 \\
\hline
BRANCH & out & B型跳转信号 \\
\hline
JUMP & out & J型跳转信号 \\
\hline
\end{tabular}
\end{center}
\end{table}

%----------------------------------------------------------------------------------------
%	SECTION 4
%----------------------------------------------------------------------------------------

\section{RegisterCluster}

寄存器堆,存放各个寄存器的值,用于读写寄存器。写寄存器为时序逻辑,读寄存器为组合逻辑。

\begin{table}[H]
\begin{center}
\renewcommand{\arraystretch}{1.3}
\small
\caption{RegisterCluster Interface}
\label{tab:treatments}
\begin{tabular}{|p{3cm}<{\centering}|p{1.4cm}<{\centering}|p{7cm}<{\centering}|}
\hline
接口名称 & 类型 & 说明 \\
\hline
clk & in & 时钟信号 \\
\hline
rst & in & 复位信号 \\
\hline
SRCREG1 & in & 输入第一个源寄存器编号 \\
\hline
SRCREG2 & in & 输入第二个源寄存器编号 \\
\hline
TARGETREG & in & 输入目的寄存器编号 \\
\hline
REGWRITE & in & 寄存器写使能 \\
\hline
WRITEDATA & in & 写回数据 \\
\hline
REGDATA1 & out & 输出第一个源寄存器值 \\
\hline
REGDATA2 & out &  输出第二个源寄存器值\\
\hline
\end{tabular}
\end{center}
\end{table}

%----------------------------------------------------------------------------------------
%	SECTION 5
%----------------------------------------------------------------------------------------

\section{PC}

计算下一周期PC值的模块,组合逻辑。根据跳转信号选择下一周期的跳转地址。为了兼顾B指令、J指令的各种情况,输入多个可能的跳转地址,包括旁路信号。

\begin{table}[H]
\begin{center}
\renewcommand{\arraystretch}{1.3}
\small
\caption{PC Interface}
\label{tab:treatments}
\begin{tabular}{|p{3cm}<{\centering}|p{1.4cm}<{\centering}|p{7cm}<{\centering}|}
\hline
接口名称 & 类型 & 说明 \\
\hline
CURRENTPC & in & 当前PC \\
\hline
LASTPC & in & 上一周期的PC \\
\hline
EXTENDIMM & in & B指令的偏移值 \\
\hline
BRANCH & in & B指令跳转类型 \\
\hline
JUMP & in & J指令跳转 \\
\hline
REG1data & in & 寄存器值,对B指令可用于判断是否跳转,对J指令指定跳转地址 \\
\hline
ALURES1 & in & ALU阶段旁路值 \\
\hline
ALURES2 & in & MEM阶段保存的ALU结果 \\
\hline
MEMDATA & in &  MEM阶段读取的内存值 \\
\hline
FORWARD & in & 数据冲突信号 \\
\hline
NEXTPC & out & 下一指令PC值 \\
\hline
IF$\_$FLUSH & out & 清除延迟槽信号 \\
\hline
\end{tabular}
\end{center}
\end{table}

%----------------------------------------------------------------------------------------
%	SECTION 6
%----------------------------------------------------------------------------------------

\section{ID$\_$EX$\_$REGISTER}

ID$\_$EXE段间锁存器,时序逻辑。传递信号,还需根据FORWARD UNIT模块的插气泡信号暂停流水线。

\begin{table}[H]
\begin{center}
\renewcommand{\arraystretch}{1.3}
\small
\caption{ID$\_$EX$\_$REGISTER Interface}
\label{tab:treatments}
\begin{tabular}{|p{3cm}<{\centering}|p{1.4cm}<{\centering}|p{7cm}<{\centering}|}
\hline
接口名称 & 类型 & 说明 \\
\hline
clk & in & 时钟信号 \\
\hline
CONTROLSTOP & in & 暂停流水线信号 \\
\hline
IN$\_$MEMTOREG & in & 输入读内存信号 \\
\hline
IN$\_$REGWRITE & in & 输入写寄存器信号 \\
\hline
IN$\_$MEMWRITE & in & 输入写内存信号 \\
\hline
IN$\_$ALUSRCA & in & 输入ALU第一个操作数选择信号 \\
\hline
IN$\_$ALUSRCB & in & 输入ALU第二个操作数选择信号 \\
\hline
IN$\_$ALUOP & in & 输入ALU操作符 \\
\hline
IN$\_$FORWARDA & in & 输入数据旁路选择信号A \\
\hline
IN$\_$FORWARDB & in & 输入数据旁路选择信号B \\
\hline
IN$\_$EXTENDIMM & in & 输入扩展的立即数 \\
\hline
IN$\_$TARGETREG & in & 输入写回寄存器编号 \\
\hline
IN$\_$REGDATA1 & in & 输入第一个寄存器值 \\
\hline
IN$\_$REGDATA2 & in & 输入第二个寄存器值 \\
\hline
IN$\_$PC & in & 输入PC值 \\
\hline
OUT$\_$MEMTOREG & out & 输出读内存信号 \\
\hline
OUT$\_$REGWRITE & out & 输出写寄存器信号 \\
\hline
OUT$\_$MEMWRITE & out & 输出写内存信号 \\
\hline
OUT$\_$ALUSRCA & out & 输出ALU第一个操作数选择信号 \\
\hline
OUT$\_$ALUSRCB & out & 输出ALU第二个操作数选择信号 \\
\hline
OUT$\_$ALUOP & out & 输出ALU运算符  \\
\hline
OUT$\_$FORWARDA & out & 输出数据旁路选择信号A \\
\hline
OUT$\_$FORWARDB & out & 输出数据旁路选择信号B \\
\hline
OUT$\_$EXTENDIMM & out & 输出扩展的立即数 \\
\hline
OUT$\_$TARGETREG & out & 输出写回寄存器编号 \\
\hline
OUT$\_$REGDATA1 & out & 输出第一个寄存器值 \\
\hline
OUT$\_$REGDATA2 & out & 输出第二个寄存器值 \\
\hline
OUT$\_$PC & out & 输出PC值 \\
\hline
\end{tabular}
\end{center}
\end{table}

%----------------------------------------------------------------------------------------
%	SECTION 7
%----------------------------------------------------------------------------------------

\section{ALU}

ALU运算器,组合逻辑。用于EXE阶段进行运算。

\begin{table}[H]
\begin{center}
\renewcommand{\arraystretch}{1.3}
\small
\caption{ALU Interface}
\label{tab:treatments}
\begin{tabular}{|p{3cm}<{\centering}|p{1.4cm}<{\centering}|p{7cm}<{\centering}|}
\hline
接口名称 & 类型 & 说明 \\
\hline
REGDATA1 & in & 第一个寄存器的值 \\
\hline
REGDATA2 & in & 第二个寄存器的值 \\
\hline
EXTENDIMM & in & 扩展的立即数 \\
\hline
LASTALU & in & 上一条指令ALU结果 \\
\hline
LASTMEM & in & 上两条指令MEM结果 \\
\hline
forwardA & in & 第一个运算数的数据冲突选择信号 \\
\hline
forwardB & in & 第二个运算数的数据冲突选择信号 \\
\hline
ALUSRCA & in & 第一个运算数的选择信号 \\
\hline
ALUSRCB & in &  第二个运算数的选择信号 \\
\hline
ALUOP & in & ALU运算符 \\
\hline
PC & in & PC值 \\
\hline
ALURES & out & ALU结果 \\
\hline
REGDATA & out & 选择的寄存器值,用于写到内存 \\
\hline
\end{tabular}
\end{center}
\end{table}

%----------------------------------------------------------------------------------------
%	SECTION 8
%----------------------------------------------------------------------------------------

\section{EX$\_$MEM$\_$REGISTER}

EXE$\_$MEM段间锁存器,时序逻辑。

\begin{table}[H]
\begin{center}
\renewcommand{\arraystretch}{1.3}
\small
\caption{EX$\_$MEM$\_$REGISTER Interface}
\label{tab:treatments}
\begin{tabular}{|p{3cm}<{\centering}|p{1.4cm}<{\centering}|p{7cm}<{\centering}|}
\hline
接口名称 & 类型 & 说明 \\
\hline
clk & in & 时钟信号 \\
\hline
IN$\_$MEMTOREG & in & 输入读内存信号 \\
\hline
IN$\_$REGWRITE & in & 输入写寄存器信号 \\
\hline
IN$\_$MEMWRITE & in & 输入写内存信号 \\
\hline
IN$\_$ALURES & in & 输入ALU计算结果 \\
\hline
IN$\_$REGDATA & in & 输入MEM写入数据 \\
\hline
IN$\_$TARGETREG & in & 输入目的寄存器编号 \\
\hline
OUT$\_$MEMTOREG & out & 输出读内存信号 \\
\hline
OUT$\_$REGWRITE & out & 输出写寄存器信号 \\
\hline
OUT$\_$MEMWRITE & out & 输出写内存信号 \\
\hline
OUT$\_$ALURES & out & 输出ALU计算结果 \\
\hline
OUT$\_$REGDATA & out & 输出MEM写入数据 \\
\hline
OUT$\_$TARGETREG & out & 输出目的寄存器编号 \\
\hline
\end{tabular}
\end{center}
\end{table}

%----------------------------------------------------------------------------------------
%	SECTION 9
%----------------------------------------------------------------------------------------

\section{MEM}

Ram1读写内存,时序逻辑。此模块中还需根据读写地址判断是否为写入指令、读写串口、键盘vga读写等等。

\begin{table}[H]
\begin{center}
\renewcommand{\arraystretch}{1.3}
\small
\caption{MEM Interface}
\label{tab:treatments}
\begin{tabular}{|p{3cm}<{\centering}|p{1.4cm}<{\centering}|p{7cm}<{\centering}|}
\hline
接口名称 & 类型 & 说明 \\
\hline
clk & in & \\
\hline
Ram1Data & inout & 第一个寄存器的值 \\
\hline
Ram1Addr & out & 第二个寄存器的值 \\
\hline
Ram1OE & out & 扩展的立即数 \\
\hline
Ram1WE & out & 上一条指令ALU结果 \\
\hline
Ram1EN & out & 上两条指令MEM结果 \\
\hline
data$\_$ready & in & 第一个运算数的数据冲突选择信号 \\
\hline
tbre & in & 第二个运算数的数据冲突选择信号 \\
\hline
tsre & in & 第一个运算数的选择信号 \\
\hline
rdn & out &  第二个运算数的选择信号 \\
\hline
wrn & out & ALU运算符 \\
\hline
uart$\_$data$\_$ready & in & PC值 \\
\hline
uart$\_$tbre & in & ALU结果 \\
\hline
uart$\_$tsre & in & 选择的寄存器值,用于写到内存 \\
\hline
uart$\_$rdn & out & \\
\hline
uart$\_$wrn & out & \\
\hline
MemData & in & \\
\hline
MemAddr & in & \\
\hline
MemWE & in & \\
\hline
MemRE & in & \\
\hline
IFWE & out & \\
\hline 
IFData & out & \\
\hline
IFAddr & out & \\
\hline
\end{tabular}
\end{center}
\end{table}


%----------------------------------------------------------------------------------------
%	SECTION 10
%----------------------------------------------------------------------------------------

\section{MEM$\_$WB$\_$REGISTER}

MEM$\_$WE段间锁存器,时序逻辑。

\begin{table}[H]
\begin{center}
\renewcommand{\arraystretch}{1.3}
\small
\caption{MEM$\_$WB$\_$REGISTER Interface}
\label{tab:treatments}
\begin{tabular}{|p{3cm}<{\centering}|p{1.4cm}<{\centering}|p{7cm}<{\centering}|}
\hline
接口名称 & 类型 & 说明 \\
\hline
clk & in & 时钟信号 \\
\hline
IN$\_$MEMTOREG & in & 输入内存数据写入寄存器信号 \\
\hline
IN$\_$REGWRITE & in & 输入写寄存器信号 \\
\hline
ALURES & in & 输入ALU计算结果 \\
\hline
IN$\_$TARGETREG & in & 输入目的寄存器编号 \\
\hline
READDATA & in & 内存读出的数据 \\
\hline
OUT$\_$REGWRITE & out & 输出写寄存器信号 \\
\hline
OUT$\_$TARGETREG & out & 输出目的寄存器编号 \\
\hline
WRITEDATA & out & 写回数据 \\
\hline
\end{tabular}
\end{center}
\end{table}

%----------------------------------------------------------------------------------------
%	SECTION 11
%----------------------------------------------------------------------------------------

\section{FORFARD UNIT}

数据冲突检测模块,组合逻辑。根据当前ID阶段指令的源寄存器编号与EXE、MEM阶段的目的寄存器编号是否相等及写回寄存器、读写内存等信号判断是否发生数据冲突及是否需要插气泡。

\begin{table}[H]
\begin{center}
\renewcommand{\arraystretch}{1.3}
\small
\caption{MEM$\_$WB$\_$REGISTER Interface}
\label{tab:treatments}
\begin{tabular}{|p{3cm}<{\centering}|p{1.4cm}<{\centering}|p{7cm}<{\centering}|}
\hline
接口名称 & 类型 & 说明 \\
\hline
SRCREG1 & in & ID阶段源寄存器1编号 \\
\hline
SRCREG2 & in & ID阶段源寄存器2编号 \\
\hline
TARGETREGALU & in & ALU阶段目的寄存器编号 \\
\hline
TARGETREGMEM & in & MEM阶段目的寄存器编号 \\
\hline
REGWRITEALU & in & ALU阶段写回寄存器信号 \\
\hline
MEMTOREGALU & in & ALU阶段读内存信号 \\
\hline
REGWRITEMEM & in & MEM阶段写回寄存器信号 \\
\hline
MEMTOREGMEM & in & MEM阶段读内存信号 \\
\hline
FORWARDA & out & 当前指令ALU操作数1数据冲突选择信号 \\
\hline
FORWARDB & out & 当前指令ALU操作数2数据冲突选择信号\\
\hline
FORWARD & out & 当前指令跳转时数据冲突选择信号\\
\hline
PCSTOP & out & 插入气泡\\
\hline
IFIDSTOP & out & 插入气泡\\
\hline
CONTROLSTOP & out & 插入气泡\\
\hline
\end{tabular}
\end{center}
\end{table}

%----------------------------------------------------------------------------------------
%	SECTION 12
%----------------------------------------------------------------------------------------

\section{keyboard}

用于键盘读入,通过状态机来接收通码、断码,并做奇偶校验。将得到的键盘输入输出到keyboardCpuRam模块进行处理。

\begin{table}[H]
\begin{center}
\renewcommand{\arraystretch}{1.3}
\small
\caption{keyboard Interface}
\label{tab:treatments}
\begin{tabular}{|p{3cm}<{\centering}|p{1.4cm}<{\centering}|p{7cm}<{\centering}|}
\hline
接口名称 & 类型 & 说明 \\
\hline
datain & in & 键盘输入数据 \\
\hline
clkin & in & 键盘时钟 \\
\hline
fclk & in & 50M时钟 \\
\hline
interrupt & out & ESC中断信号 \\
\hline
interruptC & out & Control C中断信号 \\
\hline
keyboard$\_$out & out & 键盘输入结束信号 \\
\hline
shiftstate & out & shift组合键 \\
\hline
ctrlstate & out & ctrl组合键\\
\hline
word & out & 键盘数据\\
\hline
\end{tabular}
\end{center}
\end{table}

%----------------------------------------------------------------------------------------
%	SECTION 13
%----------------------------------------------------------------------------------------

\section{keyboardCpuRam}

模拟term程序的输入部分,通过键盘的摁键设置状态机,实现A、G、U、D、R五条term程序指令,并发送给CPU模块。

\begin{table}[H]
\begin{center}
\renewcommand{\arraystretch}{1.3}
\small
\caption{keyboardCpuRam Interface}
\label{tab:treatments}
\begin{tabular}{|p{3cm}<{\centering}|p{1.4cm}<{\centering}|p{7cm}<{\centering}|}
\hline
接口名称 & 类型 & 说明 \\
\hline
clk50 & in & 50M时钟 \\
\hline
KeyBoardData & in & 键盘输入到状态机的数据 \\
\hline
DataReady & in & 键盘输入至状态机的通知 \\
\hline
CpuDataReady & out & 状态机输入到主机 \\
\hline
DataPath & inout & 数据总线 \\
\hline
CpuRdn & in & 主机给状态机读信号 \\
\hline
CpuEnable & in & 主机给状态机是否能写\\
\hline
RamData & out & 将acsii字符写入ram,用于VGA显示\\
\hline
RamDataReady & out & dataready给ram \\
\hline
RamRdn & in & ram给状态机的通知 \\
\hline
 shiftstat & in & shift组合键信号 \\
 \hline
 ctrlstate & in & ctrl组合键信号 \\
 \hline
\end{tabular}
\end{center}
\end{table}

%----------------------------------------------------------------------------------------
%	SECTION 14
%----------------------------------------------------------------------------------------

\section{keyboardstatemachine}

键盘输入部分的top模块。用于与CPU和VGA进行通信,同时访问底层keyboard模块数据,保证数据的正确性。

\begin{table}[H]
\begin{center}
\renewcommand{\arraystretch}{1.3}
\small
\caption{keyboardstatemachine Interface}
\label{tab:treatments}
\begin{tabular}{|p{3cm}<{\centering}|p{1.4cm}<{\centering}|p{7cm}<{\centering}|}
\hline
接口名称 & 类型 & 说明 \\
\hline
datain & in & 键盘输入数据 \\
\hline
clkin & in & 键盘时钟 \\
\hline
fclk & in & 50M时钟 \\
\hline
interrupt & out & ESC中断信号 \\
\hline
interruptC & out & Control C中断信号 \\
\hline
rdn & in & 主机给状态机的读信号 \\
\hline
CpuDataReady & out & 键盘给主机可读信号 \\
\hline
showData & inout & 键盘给CPU的数据 \\
\hline
KBData & out & 键盘给VGA的数据\\
\hline
KBdata$\_$ready & out & 键盘给VGA可读信号\\
\hline
KBrdn & in & VGA给键盘读使能 \\
\hline
\end{tabular}
\end{center}
\end{table}

%----------------------------------------------------------------------------------------
%	SECTION 15
%----------------------------------------------------------------------------------------

\section{compiler}

汇编代码转为机器码的模块。当term执行A指令时,用户可以输入汇编码,键盘模块将整句的汇编码保存在缓冲区中,等待回车到来后将输入的汇编码转为机器码,并发送到CPU以写入RAM2.

\begin{table}[H]
\begin{center}
\renewcommand{\arraystretch}{1.3}
\small
\caption{compiler Interface}
\label{tab:treatments}
\begin{tabular}{|p{3cm}<{\centering}|p{1.4cm}<{\centering}|p{7cm}<{\centering}|}
\hline
接口名称 & 类型 & 说明 \\
\hline
sendbuffer & in & 输入指令缓冲区 \\
\hline
instruc & out & 16位指令机器码表示 \\
\hline
\end{tabular}
\end{center}
\end{table}



