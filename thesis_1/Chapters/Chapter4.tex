% Chapter Template

\chapter{模块划分与接口设计} % Main chapter title

\label{Chapter4} % Change X to a consecutive number; for referencing this chapter elsewhere, use \ref{ChapterX}
本章介绍实验代码的模块划分,以及各个模块的接口设计。

%----------------------------------------------------------------------------------------
%	SECTION 1
%----------------------------------------------------------------------------------------

\section{IFetch}
读写指令模块,控制指令寄存器RAM2的读写操作,时序逻辑。同时根据输入PC和暂停流水线信号输出下一条顺序执行指令的PC值。

\begin{center}
\renewcommand{\arraystretch}{1.3}
\small
\begin{longtable}{|p{3cm}<{\centering}|p{1.4cm}<{\centering}|p{7cm}<{\centering}|}
\caption{IFetch Interface}
\label{tab:treatments}\\
\hline
接口名称 & 类型 & 功能 \\\hline
clk & in & 输入时钟 \\\hline
Ram2Addr & out & Ram2地址线\\\hline
Ram2Data & inout &  Ram2数据线\\\hline
PC & in & 输入PC \\\hline
PCInc & out & 输出PC \\\hline
IR & out & 输出指令 \\\hline
PCStop & in & 插入气泡信号\\\hline
IFWE & in & Ram2写信号\\\hline
IFData & in &  写入Ram2的数据 \\\hline
IFAddr & in & 写入Ram2的地址\\\hline
Ram2EN & out & Ram2的EN使能\\\hline
Ram2WE & out & Ram2的WE使能\\\hline
Ram2OE & out & Ram2的OE使能\\\hline
\end{longtable}
\end{center}

%----------------------------------------------------------------------------------------
%	SECTION 2
%----------------------------------------------------------------------------------------

\section{IF$\_$ID$\_$REGISTER}

IF$\_$ID段间锁存器,时序逻辑。除了传递信号外,还需处理中断,通过硬件方式保存现场并跳到中断处理程序。另外,还需要根据Forward Unit模块产生的插气泡信号和MEM产生的写RAM2信号进行停止流水线操作。

\begin{center}
\renewcommand{\arraystretch}{1.3}
\small
\begin{longtable}{|p{3cm}<{\centering}|p{1.4cm}<{\centering}|p{7cm}<{\centering}|}
\caption{IF$\_$ID$\_$REGISTER Interface}
\label{tab:treatments}\\
\hline
接口名称 & 类型 & 说明 \\
\hline
clk & in & 时钟信号 \\
\hline
INPC & in & 输入PC值 \\
\hline
INIR & inout & 输入指令值 \\
\hline
INTERRUPT & in & 中断信号,包括四种中断 \\
\hline
IFIDSTOP & in & 插入气泡信号 \\
\hline
IF$\_$FLUSH & in & 跳转指令清除延迟槽信号 \\
\hline
IF$\_$WRITE$\_$RAM2 & in & Ram2写指令,暂停流水线 \\
\hline
OUTPC & out & 输出PC值 \\
\hline
OUTIR & out &  输出指令值\\
\hline
\end{longtable}
\end{center}

%----------------------------------------------------------------------------------------
%	SECTION 3
%----------------------------------------------------------------------------------------

\section{Controller}

CPU控制器,组合逻辑。译码指令来产生控制信号。

\begin{center}
\renewcommand{\arraystretch}{1.3}
\small
\begin{longtable}{|p{3cm}<{\centering}|p{1.4cm}<{\centering}|p{7cm}<{\centering}|}
\caption{Controller Interface}
\label{tab:treatments}\\
\hline
接口名称 & 类型 & 说明 \\
\hline
INSTRUCTION & in & 需要译码的指令 \\
\hline
SRCREG1 & out & 第一个源寄存器编号 \\
\hline
SRCREG2 & out & 第二个源寄存器编号\\
\hline
TARGETREG & out & 目的寄存器编号 \\
\hline
EXTENDIMM & out & 扩展后的立即数 \\
\hline
ALUOP & out & ALU操作符 \\
\hline
ALUSRCA & out & ALU第一个操作数选择信号 \\
\hline
ALUSRCB & out & ALU第二个操作数选择信号 \\
\hline
MEMTOREG & out &  读内存信号 \\
\hline
REGWRITE & out & 写寄存器信号 \\
\hline
MEMWRITE & out & 写内存信号 \\
\hline
BRANCH & out & B型跳转信号 \\
\hline
JUMP & out & J型跳转信号 \\
\hline
\end{longtable}
\end{center}

%----------------------------------------------------------------------------------------
%	SECTION 4
%----------------------------------------------------------------------------------------

\section{RegisterCluster}

寄存器堆,存放各个寄存器的值,用于读写寄存器。写寄存器为时序逻辑,读寄存器为组合逻辑。

\begin{center}
\renewcommand{\arraystretch}{1.3}
\small
\begin{longtable}{|p{3cm}<{\centering}|p{1.4cm}<{\centering}|p{7cm}<{\centering}|}
\caption{RegisterCluster Interface}
\label{tab:treatments}\\
\hline
接口名称 & 类型 & 说明 \\
\hline
clk & in & 时钟信号 \\
\hline
rst & in & 复位信号 \\
\hline
SRCREG1 & in & 输入第一个源寄存器编号 \\
\hline
SRCREG2 & in & 输入第二个源寄存器编号 \\
\hline
TARGETREG & in & 输入目的寄存器编号 \\
\hline
REGWRITE & in & 寄存器写使能 \\
\hline
WRITEDATA & in & 写回数据 \\
\hline
REGDATA1 & out & 输出第一个源寄存器值 \\
\hline
REGDATA2 & out &  输出第二个源寄存器值\\
\hline
\end{longtable}
\end{center}

%----------------------------------------------------------------------------------------
%	SECTION 5
%----------------------------------------------------------------------------------------

\section{PC}

计算下一周期PC值的模块,组合逻辑。根据跳转信号判断和输出下一周期的跳转地址。为了兼顾B指令、J指令的各种情况,输入多个可能的跳转地址,包括旁路信号。

\begin{center}
\renewcommand{\arraystretch}{1.3}
\small
\begin{longtable}{|p{3cm}<{\centering}|p{1.4cm}<{\centering}|p{7cm}<{\centering}|}
\caption{PC Interface}
\label{tab:treatments}\\
\hline
接口名称 & 类型 & 说明 \\
\hline
CURRENTPC & in & 当前PC \\
\hline
LASTPC & in & 上一周期的PC \\
\hline
EXTENDIMM & in & B指令的偏移值 \\
\hline
BRANCH & in & B指令跳转类型 \\
\hline
JUMP & in & J指令跳转 \\
\hline
REG1data & in & 寄存器值,对B指令可用于判断是否跳转,对J指令指定跳转地址 \\
\hline
ALURES1 & in & ALU阶段旁路值 \\
\hline
ALURES2 & in & MEM阶段保存的ALU结果 \\
\hline
MEMDATA & in &  MEM阶段读取的内存值 \\
\hline
FORWARD & in & 数据冲突信号 \\
\hline
NEXTPC & out & 下一指令PC值 \\
\hline
IF$\_$FLUSH & out & 清除延迟槽信号 \\
\hline
\end{longtable}
\end{center}
%\end{table}

%----------------------------------------------------------------------------------------
%	SECTION 6
%----------------------------------------------------------------------------------------

\section{ID$\_$EX$\_$REGISTER}

ID$\_$EXE段间锁存器,时序逻辑。传递信号,还需根据FORWARD UNIT模块的插气泡信号暂停流水线。

\begin{center}
\renewcommand{\arraystretch}{1.3}
\small
\begin{longtable}{|p{3cm}<{\centering}|p{1.4cm}<{\centering}|p{7cm}<{\centering}|}
\caption{ID$\_$EX$\_$REGISTER Interface}
\label{tab:treatments}\\
\hline
接口名称 & 类型 & 说明 \\
\hline
clk & in & 时钟信号 \\
\hline
CONTROLSTOP & in & 暂停流水线信号 \\
\hline
IN$\_$MEMTOREG & in & 输入读内存信号 \\
\hline
IN$\_$REGWRITE & in & 输入写寄存器信号 \\
\hline
IN$\_$MEMWRITE & in & 输入写内存信号 \\
\hline
IN$\_$ALUSRCA & in & 输入ALU第一个操作数选择信号 \\
\hline
IN$\_$ALUSRCB & in & 输入ALU第二个操作数选择信号 \\
\hline
IN$\_$ALUOP & in & 输入ALU操作符 \\
\hline
IN$\_$FORWARDA & in & 输入数据旁路选择信号A \\
\hline
IN$\_$FORWARDB & in & 输入数据旁路选择信号B \\
\hline
IN$\_$EXTENDIMM & in & 输入扩展的立即数 \\
\hline
IN$\_$TARGETREG & in & 输入写回寄存器编号 \\
\hline
IN$\_$REGDATA1 & in & 输入第一个寄存器值 \\
\hline
IN$\_$REGDATA2 & in & 输入第二个寄存器值 \\
\hline
IN$\_$PC & in & 输入PC值 \\
\hline
OUT$\_$MEMTOREG & out & 输出读内存信号 \\
\hline
OUT$\_$REGWRITE & out & 输出写寄存器信号 \\
\hline
OUT$\_$MEMWRITE & out & 输出写内存信号 \\
\hline
OUT$\_$ALUSRCA & out & 输出ALU第一个操作数选择信号 \\
\hline
OUT$\_$ALUSRCB & out & 输出ALU第二个操作数选择信号 \\
\hline
OUT$\_$ALUOP & out & 输出ALU运算符  \\
\hline
OUT$\_$FORWARDA & out & 输出数据旁路选择信号A \\
\hline
OUT$\_$FORWARDB & out & 输出数据旁路选择信号B \\
\hline
OUT$\_$EXTENDIMM & out & 输出扩展的立即数 \\
\hline
OUT$\_$TARGETREG & out & 输出写回寄存器编号 \\
\hline
OUT$\_$REGDATA1 & out & 输出第一个寄存器值 \\
\hline
OUT$\_$REGDATA2 & out & 输出第二个寄存器值 \\
\hline
OUT$\_$PC & out & 输出PC值 \\
\hline
\end{longtable}
\end{center}

%----------------------------------------------------------------------------------------
%	SECTION 7
%----------------------------------------------------------------------------------------

\section{ALU}

ALU运算器,组合逻辑。用于EXE阶段进行运算。

\begin{center}
\renewcommand{\arraystretch}{1.3}
\small
\begin{longtable}{|p{3cm}<{\centering}|p{1.4cm}<{\centering}|p{7cm}<{\centering}|}
\caption{ALU Interface}
\label{tab:treatments}\\
\hline
接口名称 & 类型 & 说明 \\
\hline
REGDATA1 & in & 第一个寄存器的值 \\
\hline
REGDATA2 & in & 第二个寄存器的值 \\
\hline
EXTENDIMM & in & 扩展的立即数 \\
\hline
LASTALU & in & 上一条指令ALU结果 \\
\hline
LASTMEM & in & 上两条指令MEM结果 \\
\hline
forwardA & in & 第一个运算数的数据冲突选择信号 \\
\hline
forwardB & in & 第二个运算数的数据冲突选择信号 \\
\hline
ALUSRCA & in & 第一个运算数的选择信号 \\
\hline
ALUSRCB & in &  第二个运算数的选择信号 \\
\hline
ALUOP & in & ALU运算符 \\
\hline
PC & in & PC值 \\
\hline
ALURES & out & ALU结果 \\
\hline
REGDATA & out & 选择的寄存器值,用于写到内存 \\
\hline
\end{longtable}
\end{center}

%----------------------------------------------------------------------------------------
%	SECTION 8
%----------------------------------------------------------------------------------------

\section{EX$\_$MEM$\_$REGISTER}

EXE$\_$MEM段间锁存器,时序逻辑。

\begin{center}
\renewcommand{\arraystretch}{1.3}
\small
\begin{longtable}{|p{3cm}<{\centering}|p{1.4cm}<{\centering}|p{7cm}<{\centering}|}
\caption{EX$\_$MEM$\_$REGISTER Interface}
\label{tab:treatments}\\
\hline
接口名称 & 类型 & 说明 \\
\hline
clk & in & 时钟信号 \\
\hline
IN$\_$MEMTOREG & in & 输入读内存信号 \\
\hline
IN$\_$REGWRITE & in & 输入写寄存器信号 \\
\hline
IN$\_$MEMWRITE & in & 输入写内存信号 \\
\hline
IN$\_$ALURES & in & 输入ALU计算结果 \\
\hline
IN$\_$REGDATA & in & 输入MEM写入数据 \\
\hline
IN$\_$TARGETREG & in & 输入目的寄存器编号 \\
\hline
OUT$\_$MEMTOREG & out & 输出读内存信号 \\
\hline
OUT$\_$REGWRITE & out & 输出写寄存器信号 \\
\hline
OUT$\_$MEMWRITE & out & 输出写内存信号 \\
\hline
OUT$\_$ALURES & out & 输出ALU计算结果 \\
\hline
OUT$\_$REGDATA & out & 输出MEM写入数据 \\
\hline
OUT$\_$TARGETREG & out & 输出目的寄存器编号 \\
\hline
\end{longtable}
\end{center}

%----------------------------------------------------------------------------------------
%	SECTION 9
%----------------------------------------------------------------------------------------

\section{MEM}

访存阶段模块,控制数据寄存器Ram1的读写,时序逻辑。此模块中还需根据读写地址判断是否为写入指令、读写串口、键盘vga读写等等。

\begin{center}
\renewcommand{\arraystretch}{1.3}
\small
\begin{longtable}{|p{3cm}<{\centering}|p{1.4cm}<{\centering}|p{7cm}<{\centering}|}
\caption{MEM Interface}
\label{tab:treatments}\\
\hline
接口名称 & 类型 & 说明 \\
\hline
clk & in & 输入时钟\\
\hline
Ram1Data & inout & Ram1数据线 \\
\hline
Ram1Addr & out & Ram1地址线 \\
\hline
Ram1OE & out & Ram1的OE使能\\
\hline
Ram1WE & out & Ram1的WE使能 \\
\hline
Ram1EN & out & Ram1的EN使能 \\
\hline
data$\_$ready & in & 键盘的data\_ready信号 \\
\hline
tbre & in & VGA模块的tbre信号 \\
\hline
tsre & in & VGA模块的tsre信号 \\
\hline
rdn & out &  给键盘的rdn控制信号 \\
\hline
wrn & out & 给VGA模块的wrn控制信号 \\
\hline
uart$\_$data$\_$ready & in & 串口的data\_ready信号 \\
\hline
uart$\_$tbre & in & 串口的tbre信号 \\
\hline
uart$\_$tsre & in & 串口的tsre信号 \\
\hline
uart$\_$rdn & out & 串口的rdn控制信号 \\
\hline
uart$\_$wrn & out & 串口的wrn控制信号 \\
\hline
MemData & in & 输入的数据线\\
\hline
MemAddr & in & 输入的地址线\\
\hline
MemWE & in & 写信号\\
\hline
MemRE & in & 读信号\\
\hline
IFWE & out & 控制是否写Ram2的信号\\
\hline 
IFData & out & 给Ram2的数据\\
\hline
IFAddr & out & 给Ram2的地址\\
\hline
\end{longtable}
\end{center}


%----------------------------------------------------------------------------------------
%	SECTION 10
%----------------------------------------------------------------------------------------

\section{MEM$\_$WB$\_$REGISTER}

MEM$\_$WE段间锁存器,时序逻辑。

\begin{center}
\renewcommand{\arraystretch}{1.3}
\small
\begin{longtable}{|p{3cm}<{\centering}|p{1.4cm}<{\centering}|p{7cm}<{\centering}|}
\caption{MEM$\_$WB$\_$REGISTER Interface}
\label{tab:treatments}\\
\hline
接口名称 & 类型 & 说明 \\
\hline
clk & in & 时钟信号 \\
\hline
IN$\_$MEMTOREG & in & 输入内存数据写入寄存器信号 \\
\hline
IN$\_$REGWRITE & in & 输入写寄存器信号 \\
\hline
ALURES & in & 输入ALU计算结果 \\
\hline
IN$\_$TARGETREG & in & 输入目的寄存器编号 \\
\hline
READDATA & in & 内存读出的数据 \\
\hline
OUT$\_$REGWRITE & out & 输出写寄存器信号 \\
\hline
OUT$\_$TARGETREG & out & 输出目的寄存器编号 \\
\hline
WRITEDATA & out & 写回数据 \\
\hline
\end{longtable}
\end{center}

%----------------------------------------------------------------------------------------
%	SECTION 11
%----------------------------------------------------------------------------------------

\section{FORFARD\_UNIT}

数据冲突检测模块,组合逻辑。根据当前ID阶段指令的源寄存器编号与EXE、MEM阶段的目的寄存器编号是否相等及写回寄存器、读写内存等信号判断是否发生数据冲突及是否需要插气泡。

\begin{center}
\renewcommand{\arraystretch}{1.3}
\small
\begin{longtable}{|p{3cm}<{\centering}|p{1.4cm}<{\centering}|p{7cm}<{\centering}|}
\caption{FORFARD\_UNIT Interface}
\label{tab:treatments}\\
\hline
接口名称 & 类型 & 说明 \\
\hline
SRCREG1 & in & ID阶段源寄存器1编号 \\
\hline
SRCREG2 & in & ID阶段源寄存器2编号 \\
\hline
TARGETREGALU & in & ALU阶段目的寄存器编号 \\
\hline
TARGETREGMEM & in & MEM阶段目的寄存器编号 \\
\hline
REGWRITEALU & in & ALU阶段写回寄存器信号 \\
\hline
MEMTOREGALU & in & ALU阶段读内存信号 \\
\hline
REGWRITEMEM & in & MEM阶段写回寄存器信号 \\
\hline
MEMTOREGMEM & in & MEM阶段读内存信号 \\
\hline
FORWARDA & out & 当前指令ALU操作数1数据冲突选择信号 \\
\hline
FORWARDB & out & 当前指令ALU操作数2数据冲突选择信号\\
\hline
FORWARD & out & 当前指令跳转时数据冲突选择信号\\
\hline
PCSTOP & out & 插入气泡\\
\hline
IFIDSTOP & out & 插入气泡\\
\hline
CONTROLSTOP & out & 插入气泡\\
\hline
\end{longtable}
\end{center}

%----------------------------------------------------------------------------------------
%	SECTION 12
%----------------------------------------------------------------------------------------

\section{Keyboard}

用于键盘读入,通过状态机来接收通码、断码,并做奇偶校验。将得到的键盘输入输出到keyboardCpuRam模块进行处理。

\begin{center}
\renewcommand{\arraystretch}{1.3}
\small
\begin{longtable}{|p{3cm}<{\centering}|p{1.4cm}<{\centering}|p{7cm}<{\centering}|}
\caption{Keyboard Interface}
\label{tab:treatments}\\
\hline
接口名称 & 类型 & 说明 \\
\hline
datain & in & 键盘输入数据 \\
\hline
clkin & in & 键盘时钟 \\
\hline
fclk & in & 50M时钟 \\
\hline
interrupt & out & ESC中断信号 \\
\hline
interruptC & out & Control C中断信号 \\
\hline
keyboard$\_$out & out & 键盘输入结束信号 \\
\hline
shiftstate & out & shift组合键 \\
\hline
ctrlstate & out & ctrl组合键\\
\hline
word & out & 键盘数据\\
\hline
\end{longtable}
\end{center}

%----------------------------------------------------------------------------------------
%	SECTION 13
%----------------------------------------------------------------------------------------

\section{KeyboardCpuRam}

模拟term程序的输入部分,通过键盘的摁键设置状态机,实现A、G、U、D、R五条term程序指令,并发送给CPU模块。

\begin{center}
\renewcommand{\arraystretch}{1.3}
\small
\begin{longtable}{|p{3cm}<{\centering}|p{1.4cm}<{\centering}|p{7cm}<{\centering}|}
\caption{KeyboardCpuRam Interface}
\label{tab:treatments}\\
\hline
接口名称 & 类型 & 说明 \\
\hline
clk50 & in & 50M时钟 \\
\hline
KeyBoardData & in & 键盘输入到状态机的数据 \\
\hline
DataReady & in & 键盘输入至状态机的通知 \\
\hline
CpuDataReady & out & 状态机输入到主机 \\
\hline
DataPath & inout & 数据总线 \\
\hline
CpuRdn & in & 主机给状态机读信号 \\
\hline
CpuEnable & in & 主机给状态机是否能写\\
\hline
RamData & out & 将acsii字符写入ram,用于VGA显示\\
\hline
RamDataReady & out & dataready给ram \\
\hline
RamRdn & in & ram给状态机的通知 \\
\hline
 shiftstat & in & shift组合键信号 \\
 \hline
 ctrlstate & in & ctrl组合键信号 \\
 \hline
\end{longtable}
\end{center}

%----------------------------------------------------------------------------------------
%	SECTION 14
%----------------------------------------------------------------------------------------

\section{KeyboardStateMachine}

键盘输入部分的top模块。用于与CPU和VGA进行通信,同时访问底层keyboard模块数据,保证数据的正确性。

\begin{center}
\renewcommand{\arraystretch}{1.3}
\small
\begin{longtable}{|p{3cm}<{\centering}|p{1.4cm}<{\centering}|p{7cm}<{\centering}|}
\caption{KeyboardStateMachine Interface}
\label{tab:treatments}\\
\hline
接口名称 & 类型 & 说明 \\
\hline
datain & in & 键盘输入数据 \\
\hline
clkin & in & 键盘时钟 \\
\hline
fclk & in & 50M时钟 \\
\hline
interrupt & out & ESC中断信号 \\
\hline
interruptC & out & Control C中断信号 \\
\hline
rdn & in & 主机给状态机的读信号 \\
\hline
CpuDataReady & out & 键盘给主机可读信号 \\
\hline
showData & inout & 键盘给CPU的数据 \\
\hline
KBData & out & 键盘给VGA的数据\\
\hline
KBdata$\_$ready & out & 键盘给VGA可读信号\\
\hline
KBrdn & in & VGA给键盘读使能 \\
\hline
\end{longtable}
\end{center}

%----------------------------------------------------------------------------------------
%	SECTION 15
%----------------------------------------------------------------------------------------

\section{Compiler}

汇编代码转为机器码的模块。当term执行A指令时,用户可以输入汇编码,键盘模块将整句的汇编码保存在缓冲区中,等待回车到来后将输入的汇编码转为机器码,并发送到CPU以写入RAM2.

\begin{center}
\renewcommand{\arraystretch}{1.3}
\small
\begin{longtable}{|p{3cm}<{\centering}|p{1.4cm}<{\centering}|p{7cm}<{\centering}|}
\caption{Compiler Interface}
\label{tab:treatments}\\
\hline
接口名称 & 类型 & 说明 \\
\hline
sendbuffer & in & 输入指令缓冲区 \\
\hline
instruc & out & 16位指令机器码表示 \\
\hline
\end{longtable}
\end{center}

%----------------------------------------------------------------------------------------
%	SECTION 16
%----------------------------------------------------------------------------------------

\section{VGA\_Controller}

VGA控制模块,控制计算机的VGA显示功能,主要任务是将需要显示的数据存入显存(一块FPGA片内RAM),并控制VGA的显示。

\begin{center}
\renewcommand{\arraystretch}{1.3}
\small
\begin{longtable}{|p{3cm}<{\centering}|p{1.4cm}<{\centering}|p{7cm}<{\centering}|}
\caption{VGA\_Controller Interface}
\label{tab:treatments}\\
\hline
接口名称 & 类型 & 说明 \\\hline
clk50 & in & 50M时钟 \\\hline
rst & in  & 复位信号 \\\hline
DataIn & in  & 输入数据 \\\hline
data\_ready & in & VGA的data\_ready信号\\\hline
rdn & out  & VGA的rdn控制信号 \\\hline
hs & out & VGA行同步信号\\\hline
vs & out & VGA场同步信号\\\hline
r, g, b & out & VGA输出的RGB \\\hline
\end{longtable}
\end{center}

%----------------------------------------------------------------------------------------
%	SECTION 17
%----------------------------------------------------------------------------------------

\section{VGA}

VGA模块,控制VGA输出,包括从一块FPGA片内ROM中读取字符图片并显示。

\begin{center}
\renewcommand{\arraystretch}{1.3}
\small
\begin{longtable}{|p{3cm}<{\centering}|p{1.4cm}<{\centering}|p{7cm}<{\centering}|}
\caption{VGA\_Controller Interface}
\label{tab:treatments}\\
\hline
接口名称 & 类型 & 说明 \\\hline
clk50 & in & 50M时钟 \\\hline
rst & in  & 复位信号 \\\hline
DataIn & in  & 输入数据 \\\hline
data\_ready & in & VGA的data\_ready信号\\\hline
hs & out & VGA行同步信号\\\hline
vs & out & VGA场同步信号\\\hline
r, g, b & out & VGA输出的RGB \\\hline
row & out & 字符行 \\\hline
column & out & 字符列 \\\hline
data & in & 一个像素的色彩值 \\\hline
\end{longtable}
\end{center}

%----------------------------------------------------------------------------------------
%	SECTION 18
%----------------------------------------------------------------------------------------

\section{VGA\_Core\_Ball}

屏幕保护显示模块,控制在屏幕保护阶段的VGA输出。这一部分代码由作者上一学期《数字逻辑设计》课程的课程项目经过精简后得到。

\begin{center}
\renewcommand{\arraystretch}{1.3}
\small
\begin{longtable}{|p{3cm}<{\centering}|p{1.4cm}<{\centering}|p{7cm}<{\centering}|}
\caption{VGA\_Core\_Ball Interface}
\label{tab:treatments}\\
\hline
接口名称 & 类型 & 说明 \\\hline
clk & in & 25M时钟输入 \\\hline
reset & in  & 复位信号 \\\hline
r, g, b & out & 输出的RGB \\\hline
vector\_x & in & 当前VGA输出的行 \\\hline
vector\_y & in & 当前VGA输出的列 \\\hline
num\_c & in & 小球个数 \\\hline
theta\_c & in & 小球角度 \\\hline
mtime  & in & 时间 \\\hline
\end{longtable}
\end{center}


%----------------------------------------------------------------------------------------
%	SECTION 19
%----------------------------------------------------------------------------------------

\section{Output\_Adaptor}

输出转换接口模块。VGA的显示有来自CPU和来自键盘两个不同的来源,经过这个模块的处理整合后提供给VGA控制模块。

\begin{center}
\renewcommand{\arraystretch}{1.3}
\small
\begin{longtable}{|p{3cm}<{\centering}|p{1.4cm}<{\centering}|p{7cm}<{\centering}|}
\caption{Output\_Adaptor Interface}
\label{tab:treatments}\\
\hline
接口名称 & 类型 & 说明 \\\hline
clk50 & in & 50M时钟 \\\hline
CPUData & in & CPU的数据 \\\hline
CPUwrn & in  & CPU的wrn信号 \\\hline
CPUtsre & out  & CPU的tsre信号 \\\hline
KBData & in  & 键盘的数据 \\\hline
KBdata\_ready & in & 键盘的data\_ready \\\hline
KBrdn & out & 键盘的rdn信号 \\\hline
DataOut & out & 输出的数据 \\\hline
data\_ready & out & 输出的data\_ready信号 \\\hline
rdn & in & rdn控制信号 \\\hline
\end{longtable}
\end{center}

